\documentclass[onecolumn, draftclsnofoot,10pt, compsoc]{IEEEtran}
\usepackage{graphicx}
\usepackage{url}
\usepackage{setspace}
\usepackage{array}
\usepackage{geometry}
\geometry{textheight=9.5in, textwidth=7in}

% 1. Fill in these details
\def \CapstoneTeamName{		ConnectBasket Development Team}
\def \CapstoneTeamNumber{		39}
\def \GroupMemberOne{			Kailyn Hellwege}
\def \CapstoneProjectName{		ConnectBasket}
\def \CapstoneSponsorCompany{	OSU MIME}
\def \CapstoneSponsorPerson{		Dr. Chinweike Eseonu}

% 2. Uncomment the appropriate line below so that the document type works
\def \DocType{		%Problem Statement
				%Requirements Document
				Technology Review
				%Design Document
				%Progress Report
				}
			
\newcommand{\NameSigPair}[1]{\par
\makebox[2.75in][r]{#1} \hfil 	\makebox[3.25in]{\makebox[2.25in]{\hrulefill} \hfill		\makebox[.75in]{\hrulefill}}
\par\vspace{-12pt} \textit{\tiny\noindent
\makebox[2.75in]{} \hfil		\makebox[3.25in]{\makebox[2.25in][r]{Signature} \hfill	\makebox[.75in][r]{Date}}}}
% 3. If the document is not to be signed, uncomment the RENEWcommand below
\renewcommand{\NameSigPair}[1]{#1}

%%%%%%%%%%%%%%%%%%%%%%%%%%%%%%%%%%%%%%%
\begin{document}
\begin{titlepage}
    \pagenumbering{gobble}
    \begin{singlespace}
    	\includegraphics[height=4cm]{coe_v_spot1}
        \hfill 
        % 4. If you have a logo, use this includegraphics command to put it on the coversheet.
        %\includegraphics[height=4cm]{CompanyLogo}   
        \par\vspace{.2in}
        \centering
        \scshape{
            \huge CS Capstone \DocType \par
            {\large\today}\par
            \vspace{.5in}
            \textbf{\Huge\CapstoneProjectName}\par
            \vfill
            {\large Prepared for}\par
            \Huge \CapstoneSponsorCompany\par
            \vspace{5pt}
            {\Large\NameSigPair{\CapstoneSponsorPerson}\par}
            {\large Prepared by }\par
            Group\CapstoneTeamNumber\par
            % 5. comment out the line below this one if you do not wish to name your team
            \CapstoneTeamName\par 
            \vspace{5pt}
            {\Large
                \NameSigPair{\GroupMemberOne}\par
            }
            \vspace{20pt}
        }
        \begin{abstract}
        % 6. Fill in your abstract    
        This document discusses three main components of the ConnectBasket project and compares and contrasts different technologies that can be used to implement those components of the project. This document will discuss content delivery networks, web development frameworks, and software design patterns. For each component, three different technologies will be researched, compared, and contrasted to determine which one will work best for the ConnectBasket project.
        \end{abstract}     
    \end{singlespace}
\end{titlepage}
\newpage
\pagenumbering{arabic}
\tableofcontents
% 7. uncomment this (if applicable). Consider adding a page break.
%\listoffigures
%\listoftables
\clearpage

% 8. now you write!

\section{Introduction}

\subsection{Reason for Project}
The current software system used by the Oregon State University Veterinary Hospital for communication is outdated and inefficient. In their current system of receiving messages, if an owner calls the hospital with a medical question, the receptionist would log the message into the computer, print it out, and physically walk the paper across the hospital to put in a spinning queue. When a doctor or other staff member looks at it, they would write an answer or necessary action on the paper, and physically carry the paper to the next location. Eventually, after it reaches all the necessary staff, it would be returned to the receptionist, who would then call the owner back to provide them with the information they requested. Currently, their process is very slow and manual, with lots of movement around a large hospital and unnecessary printing of papers.

\subsection{Overview of Project}
The ConnectBasket project will be a stand-alone solution that will streamline the communication process in the hospital. A receptionist will be able to receive a call and create a new message. They can then assign a category to the message and route it to the necessary staff, who will be able to add notes and reroute it to more staff members. Staff will receive notifications, by email or text, when they have a message waiting to be viewed. Staff will have profiles where they can select their notification preferences, as well as which categories of messages they want to receive. When the message returns to the receptionist, they can call back the owner and inform them of the decisions or recommendations by the staff. The whole chain of where the message has traveled to and who has seen the message or added notes will be visible. This will help the hospital track specific metrics, such as response time to calls. 

\subsection{Overview of Document}
In this document, three components of the project will be examined, looking at three different technologies for each that could be used to implement that component. ConnectBasket will be a web-based application, and a content delivery network will be useful for delivering internet content, as well as improving load times and security. ConnectBasket will also require a web development framework. Finally, a software design pattern will be helpful for developing a simple, effective user interface. 


\section{Content Delivery Networks}

\subsection{Overview}
A content delivery network, or CDN, is a group of servers that are distributed geographically that work together to quickly deliver internet content, such as images, videos, and HTML pages. Currently, most web traffic is served through CDNs, and CDN's can also help protect against some malicious attacks. There are four main reasons to use a CDN: to improve website load times, to reduce bandwidth costs, to increase content availability, and to improve website security \cite{cdn}. 

\subsection{Criteria}
For this project, security is an important factor to consider. Since the ConnectBasket website will contain patient and owner information, it is important that the website is secure. Another essential factor to consider for the ConnectBasket project is price. It is critical for the cost to stay as low as possible, so choosing a free or low cost CDN is important to the project. In addition, fast and easy setup for the CDN would be beneficial, since there is a relatively small time frame for the development of ConnectBasket.


\subsection{Potential Choices}

\subsubsection{Cloudflare}
The CDN that Cloudflare provides is designed to optimize security and performance, has multiple price levels, and is easy to set up. Cloudflare's CDN provides a flat price per month that does not depend on bandwidth. There are four pricing levels. The free version is for personal websites. The Pro version is twenty dollars per month for professional websites. The Business version is two hundred dollars per month for websites and businesses requiring advanced security and performance. Finally, the Enterprise version is for companies requiring enterprise-grade security and performance, and Cloudflare muse be contacted to determine a monthly price. In addition, setting up Cloudflare is easy, taking less than five minutes to set up a domain \cite{cloudflare}.

\subsubsection{Akamai}
Akamai serves thirty percent of internet traffic, and it optimizes networks for content delivery \cite{akamaicdn}. Akamai's CDN has three components \cite{akamaicdn}. The Aura Edge Exchange allows operators to deliver video content to customers \cite{akamaicdn}. The Aura Control System is a set of tools that allows operators to manage components such as performance and CDN security \cite{akamaicdn}. Lastly, Akamai Federeation allows operators to work with the Akamai Intelligent Platform, which is a cloud-computing platform. Akamai does not provide pricing information on their website; they must be directly contacted to get a quote \cite{akamaicdn}.  

\subsubsection{Amazon CloudFront}
The Amazon CloudFront CDN is built on the Amazon Web Service (AWS) infrastructure, with forty-four Availability zones in sixteen regions, and there are plans for further expansion. This large network provides high performance for customers all over the world. CloudFront is a highly secure CDN, and for no additional cost, all customers benefit from the protections of AWS Shield Standard. Amazon CloudFront is easily integrated and works best with other AWS services, such as Amazon Simple Storage Service and Amazon API Gateway. For pricing, there are no fixed fees or long term contracts; payment is only for the data transfer and requests used to deliver content to customers \cite{amazoncloudfront}.


\subsection{Discussion}
Cloudflare and Amazon seem to have opposite pricing systems. Cloudflare has a fixed pricing system, where the amount of traffic per month does not affect how much is paid. In contrast, for Amazon CloudFront has a variable pricing system, where payment is only for the amount of data transfer and requests. Akamai does not provide pricing information, and so a price comparison is not possible. All three have a high level of security, and so they would all be good choices for that factor.

\subsection{Conclusion}
The best CDN for this project is Cloudflare because it has a fixed cost, which means there will not be any spikes in cost in a given month. In addition, the amount of traffic to the ConnectBasket website is unknown, so a fixed cost is more reliable. Amazon CloudFront would be a good CDN, especially since it integrates well with other AWS services, but it seems to be the best choice only if other AWS services are being used, and we are not using other AWS services.

\begin{table}[h!]
\centering
\begin{tabular}{ |p{.2\linewidth}|p{.2\linewidth}|p{.2\linewidth}|p{.2\linewidth}| } 
\hline
\textbf{Criteria} & \textbf{Cloudflare} & \textbf{Akamai} & \textbf{Amazon CloudFront} \\ \hline
Cost & Fixed cost per month & Unknown & Variable cost per month \\ \hline
Difficulty of setup & Very easy, taking less than five minutes & Easy, with 24/7 support from Akamai & Easy when used with other AWS services\\ \hline
Security & Secure & Secure & Secure \\ \hline
\end{tabular}
\end{table}


\section{Web Development Frameworks}

\subsection{Overview}
A web development framework is considered to be tools and resources used by software developers to create and manage websites \cite{webdevframework}. Web development frameworks extend the capabilities of a language and provide libraries so developers do not have to start from scratch and hand-code everything. A framework may include libraries, APIs, security, and compilers, among other things \cite{framework}.

\subsection{Criteria}
One important criteria for this project for web development frameworks is that it needs to be easily scalable to many different devices and sizes of devices. This is important because the ConnectBasket website will need to be accessible from staff's desktop or laptop computers as well as a variety of smartphone devices. Another key factor is it needs to be simple as opposed to cluttered with a large number of features that will likely go unused. This project is relatively simple, and so tons of features are not necessary. Additionally, a framework that is flexible would be best. 

\subsection{Potential Choices}

\subsubsection{ASP.NET} 
ASP.NET was created by Microsoft, and it is used to create interactive and data-driven web applications\cite{asp.net}. ASP.NET works with HTML to create dynamic web pages. One advantage of ASP.NET is that it provides built-in Windows authentication, so it is secure \cite{asp}. In addition, an ASP.NET application can be written in a variety of languages, including C\# and Visual Basic \cite{asp.net}. 

\subsubsection{AngularJS}
AngularJS is a JavaScript based web application framework that is open source and currently maintained by Google \cite{angular}. It is fast for development and creates simple, dynamic websites \cite{angularjs}. AngularJS uses HTML as a template and extends HTML syntax to describe a website or application's components easily. AngularJS provides data binding capability to HTML, provides reusable components, and is unit testable, among many other features. In addition, AngularJS is available on all major browsers and smartphones. One disadvantage of AngularJS is that it is not secure, and so server side authentication is needed to keep an application secure \cite{angular}.

\subsubsection{ReactJS}
ReactJS is a JavaScript library that makes it easy to build user interfaces by creating reusable UI components\cite{react}. ReactJS involves thinking about everything as a component \cite{react}. Advantages of ReactJS include that it can be used with other frameworks, and can be used on the client side or the server side \cite{react}. One major disadvantage of ReactJS is that other tools will need to be combined with React to get a complete set of tools that is required for development \cite{react}. 

\subsection{Discussion}
AngularJS and ReactJS are both JavaScript based, while ASP.NET is very flexible because it can be written in more than one language. AngularJS is useful because it is available on all major browsers and smartphones, but a downside is that it is not secure and would still require server side authentication. However, another component of the project is user authenication, so it is not required that the chosen web development framework contains secure authentication. In contrast, ASP.NET is secure and provides built-in Windows authentication. ReactJS would be reasonable for creating reusable components, but it would possibly require additional tools for successful development. AngularJS also can easily create reusable components.

\subsection{Conclusion}
AngularJS is the best web development framework for the ConnectBasket project. The main reason for choosing AngularJS is because it is designed to quickly produce dynamic and elegant websites. It is well-documented and maintained by Google, which makes it a sensible choice. Even though it does not have user authenication, that is something that can easily be implemented using another technology. 

\begin{table}[h!]
\centering
\begin{tabular}{ |p{.2\linewidth}|p{.2\linewidth}|p{.2\linewidth}|p{.2\linewidth}| } 
\hline
\textbf{Criteria} & \textbf{ASP.NET} & \textbf{AngularJS} & \textbf{ReactJS} \\ \hline
Flexibilty & Can be written in multiple languages & Works with all major browsers and smartphones & Works with all major browsers and smartphones \\ \hline
Complexity & Moderately complex & Simple & Simple\\ \hline
Integration with other tools & Works best with other Microsoft products & Easily integrates with other tools & Easily integrates with other tools \\ \hline
\end{tabular}
\end{table}

\section{Software Design Patterns}

\subsection{Overview}
Software design patterns are a important part of any software application. One very important feature of this project is to provide a user interface that will be accepted by users who are reluctant to change. This means how users are able to interact with the data is very important and modeling a system that closely reflects their current system in a more efficient way is critical.

\subsection{Criteria}
One major factor to consider when choosing a software design pattern for this project is how well the system will be able to model data in a way that our system needs it to. Users will be inputting data into the system, which must to be able to display that information to other users. Some data will be filtered for specific users, while some data will need to be displayed as an overall summary for other users.

\subsection{Potential Choices}

\subsubsection{MVC}
MVC is a software design pattern for implementing a user interface. It separates an application into three parts known as the model, view, and controller. The model is the structure of the data in the application and how the data is represented in the application. View is the component that makes the data displayable to the user in a useful way. The controller supplies an interface between the model and view components, taking the data from the model and converting it into something that can be used by the view component to display to the user. MVC supports the development of multiple different views for one model so the same information can be displayed in multiple ways if that is important for the application. It is also a very fast development process because one person or team can develop the view component while another works on the controller and another works on the model. This design pattern requires multiple developers for it to work properly \cite{mvc}.

\subsubsection{MVP}
MVP is a software design pattern that is based on similar concepts to the MVC design pattern. It separates an application into four different components that are responsible for determining how a user can interact with the system. Those components are the view, view interface, presenter, and model. The view is responsible for determining how the user will see the information displayed while the view interface connects the view to the presenter. The presenter connects the view to the model, which is responsible for the data that is displayed by the view. A view will usually only be connected to one presenter. Testing can be done easily with this design pattern because all interactions are done through an interface. Development is sped up by the fact that multiple programmers can split up the components and work on them separately \cite{mvp}.

\subsubsection{MVVM}
MVVM is a software design pattern that is an extension of the MVC design pattern. MVVM has four components, which are model, view, controller, and view model. In the MVVM pattern, view models convert data from the model layer into something that is usable by the view layer, which keeps the code for controllers from growing too large. In MVVM, the controller no longer depends on the model, so it is much easier to test. Views are only used to present the data they are given. The view controller interacts with both the view model and view layers and the view model also interacts with the model \cite{mvvm}.

\subsection{Discussion}
MVVM and MVP are both similar to MVC, and provide some additional features compared to MVC. In contrast to the four components of MVVM and MVP, MVC only has three components which can make it a simpler and easier to use pattern. MVVM and MVP have an additional components which further modularizes the functionality of each component and makes it easier to test.

\subsection{Conclusion}
The best design pattern to use for this project is MVC. The user interface for this project will not be very complicated, so, while having additional components might improve testability, it will not be worth it for the increased complexity. Having three components also provides the option for dividing the components between the three person development team, with one team member working on each component.

\begin{table}[h!]
\centering
\begin{tabular}{ |p{.2\linewidth}|p{.2\linewidth}|p{.2\linewidth}|p{.2\linewidth}| } 
\hline
\textbf{Criteria} & \textbf{MVC} & \textbf{MVP} & \textbf{MVVM} \\ \hline
Components & Model, view, controller & View, view interface, presenter, model & Model, view, controller, view model \\ \hline
Number of components & 3 & 4 & 4 \\ \hline
\end{tabular}
\end{table}

\newpage
\bibliographystyle{ieeetr}
\bibliography{technology_review}

\end{document}
