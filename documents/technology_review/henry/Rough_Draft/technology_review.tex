\documentclass[onecolumn, draftclsnofoot,10pt, compsoc]{IEEEtran}
\usepackage{graphicx}
\usepackage{url}
\usepackage{setspace}

\usepackage{geometry}
\geometry{textheight=9.5in, textwidth=7in}

% 1. Fill in these details
\def \CapstoneTeamName{		ConnectBasket Development Team}
\def \CapstoneTeamNumber{		39}
\def \GroupMemberOne{			Henry Fowler}
\def \CapstoneProjectName{		ConnectBasket}
\def \CapstoneSponsorCompany{	OSU MIME}
\def \CapstoneSponsorPerson{		Dr. Chinweike Eseonu}

% 2. Uncomment the appropriate line below so that the document type works
\def \DocType{		%Problem Statement
				Requirements Document
				%Technology Review
				%Design Document
				%Progress Report
				}
			
\newcommand{\NameSigPair}[1]{\par
\makebox[2.75in][r]{#1} \hfil 	\makebox[3.25in]{\makebox[2.25in]{\hrulefill} \hfill		\makebox[.75in]{\hrulefill}}
\par\vspace{-12pt} \textit{\tiny\noindent
\makebox[2.75in]{} \hfil		\makebox[3.25in]{\makebox[2.25in][r]{Signature} \hfill	\makebox[.75in][r]{Date}}}}
% 3. If the document is not to be signed, uncomment the RENEWcommand below
\renewcommand{\NameSigPair}[1]{#1}

%%%%%%%%%%%%%%%%%%%%%%%%%%%%%%%%%%%%%%%
\begin{document}
\begin{titlepage}
    \pagenumbering{gobble}
    \begin{singlespace}
    	\includegraphics[height=4cm]{coe_v_spot1}
        \hfill 
        % 4. If you have a logo, use this includegraphics command to put it on the coversheet.
        %\includegraphics[height=4cm]{CompanyLogo}   
        \par\vspace{.2in}
        \centering
        \scshape{
            \huge CS Capstone \DocType \par
            {\large\today}\par
            \vspace{.5in}
            \textbf{\Huge\CapstoneProjectName}\par
            \vfill
            {\large Prepared for}\par
            \Huge \CapstoneSponsorCompany\par
            \vspace{5pt}
            {\Large\NameSigPair{\CapstoneSponsorPerson}\par}
            {\large Prepared by }\par
            Group\CapstoneTeamNumber\par
            % 5. comment out the line below this one if you do not wish to name your team
            \CapstoneTeamName\par 
            \vspace{5pt}
            {\Large
                \NameSigPair{\GroupMemberOne}\par
            }
            \vspace{20pt}
        }
        \begin{abstract}
        % 6. Fill in your abstract    
        This document describes the three main parts of the ConnectBasket project that Henry Fowler will be responsible for and compares and contrasts different technologies that can be used to implement those parts of the project.
        \end{abstract}     
    \end{singlespace}
\end{titlepage}
\newpage
\pagenumbering{arabic}
\tableofcontents
% 7. uncomment this (if applicable). Consider adding a page break.
%\listoffigures
%\listoftables
\clearpage

% 8. now you write!
\section{Server Operating System}

\subsection{Overview}
For this project there will need to be a server to host the web based services involved in the project. There are many different choices of operating systems that can run on a server and this section will analyze three different options and show that one will be the best fit for the project.

\subsection{Criteria}
The criteria that will be used to choose the best server operating system for this project will be the cost, difficulty of setup, available support, necessary downtime, and amount of maintenance needed. For this project using low cost or free technologies will be very useful in keeping the overall budget low. There is a relatively small time window for development of the project, so a fast and easy installation is very important. It will also be important to have to have a low maintenance system that will continue to work after the development team is gone.

\subsection{Potential Choices}

\subsubsection{Linux Ubuntu Server}
Ubuntu is a distribution of Linux that is very popular on desktop computers, but also has a server version that is growing in popularity /cite{ubuntu}. Unlike the very popular Red Hat Linux distribution, Ubuntu is free /cite{ubuntu}. Updates and maintenance of the code and security of Ubuntu are made by the global community as well as a paid development team /cite{ubuntu}. There is 24x7 support available that can be very important if a quick response is need to critical errors /cite{ubuntu}. Ubuntu is an easy system to setup with an installation that can be completed in as little as 15 minutes /cite{ubuntu}. Like other versions of Linux, Ubuntu Server is very easy to maintain and rarely requires reboots, making it ideal for an application that requires very little downtime /cite{ubuntu}. 

\subsubsection{Windows Server}
The Windows Server operating systems have been very popular with businesses since they first came out /cite{windows}. Many tools are provided that will work very well with Windows Server, including Microsoft SQL Server and Internet Information Services /cite{windows}. A large number of successful businesses use Windows Server, which means that there are many examples of how to use it as a part of a business /cite{windows}. Windows Server is considered by many to be the best server operating system for mission-critical systems, meaning that there is very little downtime required when using Windows Server /cite{windows}. Windows Server has built in tools to help improve power efficiency of the server that can help lower power consumption for a company /cite{windows}. A virtualization platform is also included with Windows server which can allow for easily migrating the system from one physical machine to another or allowing users to remotely access applications and run them like they were locally installed /cite{windows}.

\subsubsection{Solaris}
Solaris is a version of Unix that has been developed as a server operating system /cite{solaris}. Solaris was created by Sun and they are also responsible for supporting it /cite{solaris}. This makes Solaris a good operating system for a user who will need a lot of help in setting up their system /cite{solaris}. Solaris is based off of Unix, which is an operating system that has been around a long time and is trusted by many businesses /cite{solaris}. A problem with Solaris is that it will not run on many different hardware systems including most HP and IBM systems /cite{solaris}. Solaris is not free, although there is a version of Solaris that is more similar to Linux and can be optained for free with optional paid support /cite{solaris}.

\subsection{Discussion}
Unix, which Solaris is based off of has been around a long time compared to Windows Server and Linux Ubuntu Server, but is not currently as popular as either of the other two. Linux Ubuntu Server is completely free with optional support packages that can be purchased, in contrast to Windows Server and Solaris which both cost money. All of the operating systems provide key features that make any server operating system useful such as security, support, easy setup. It might be slightly harder to get support when using a Linux system than the other two because Linux isn't owned by anyone.

\subsection{Conclusion}
The best server operating system to use for our project is Ubuntu Server. Solaris isn't free and doesn't really provide as many features as Windows Server or Ubuntu so it is definitely not the right choice for our project. Windows Server might provide slightly more features than Ubuntu, but all we are using the server for is to host a simple web application, so things like allowing users to run desktop applications remotely aren't important for our project. For all of the important features to our project like setup, support, downtime, and maintenance, Windows Server and Ubuntu aren't that different and both would be good choices. The main difference is that Ubuntu is free, and for that reason a better choice than Windows Server for this project.

\section{Web Server Software}

\subsection{Overview}

\subsection{Criteria}

\subsection{Potential Choices}

\subsubsection{Apache HTTP}

\subsubsection{Cherokee}

\subsubsection{NGINX}

\subsection{Discussion}

\subsection{Conclusion}

\section{Database}

\subsection{Overview}

\subsection{Criteria}

\subsection{Potential Choices}

\subsubsection{MySQL}

\subsubsection{SQLite}

\subsubsection{Microsoft SQL Server}

\subsection{Discussion}

\subsection{Conclusion}

\end{document}
