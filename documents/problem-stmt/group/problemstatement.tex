\documentclass[onecolumn, draftclsnofoot,10pt, compsoc]{IEEEtran}
\usepackage{graphicx}
\usepackage{url}
\usepackage{setspace}

\usepackage{geometry}
\geometry{textheight=9.5in, textwidth=7in}

% 1. Fill in these details
\def \CapstoneTeamName{		Group 39}
\def \CapstoneTeamNumber{		39}
\def \GroupMemberOne{			Henry Fowler}
\def \GroupMemberTwo{			Kailyn Hellwege}
\def \GroupMemberThree{			Taylor Kirkpatrick}
\def \CapstoneProjectName{		Healthy Dogs! Software for Managing Pet Safety at a Veterinary Hospital}
\def \CapstoneSponsorCompany{	OSU MIME}
\def \CapstoneSponsorPerson{		C Eseonu}

% 2. Uncomment the appropriate line below so that the document type works
\def \DocType{		Problem Statement
				%Requirements Document
				%Technology Review
				%Design Document
				%Progress Report
				}
			
\newcommand{\NameSigPair}[1]{\par
\makebox[2.75in][r]{#1} \hfil 	\makebox[3.25in]{\makebox[2.25in]{\hrulefill} \hfill		\makebox[.75in]{\hrulefill}}
\par\vspace{-12pt} \textit{\tiny\noindent
\makebox[2.75in]{} \hfil		\makebox[3.25in]{\makebox[2.25in][r]{Signature} \hfill	\makebox[.75in][r]{Date}}}}
% 3. If the document is not to be signed, uncomment the RENEWcommand below
\renewcommand{\NameSigPair}[1]{#1}

%%%%%%%%%%%%%%%%%%%%%%%%%%%%%%%%%%%%%%%
\begin{document}
\begin{titlepage}
    \pagenumbering{gobble}
    \begin{singlespace}
    	\includegraphics[height=4cm]{coe_v_spot1}
        \hfill 
        % 4. If you have a logo, use this includegraphics command to put it on the coversheet.
        %\includegraphics[height=4cm]{CompanyLogo}   
        \par\vspace{.2in}
        \centering
        \scshape{
            \huge CS Capstone \DocType \par
            {\large\today}\par
            \vspace{.5in}
            \textbf{\Huge\CapstoneProjectName}\par
            \vfill
            {\large Prepared for}\par
            \Huge \CapstoneSponsorCompany\par
            \vspace{5pt}
            {\Large\NameSigPair{\CapstoneSponsorPerson}\par}
            {\large Prepared by }\par
            Group\CapstoneTeamNumber\par
            % 5. comment out the line below this one if you do not wish to name your team
            %\CapstoneTeamName\par 
            \vspace{5pt}
            {\Large
                \NameSigPair{\GroupMemberOne}\par
                \NameSigPair{\GroupMemberTwo}\par
                \NameSigPair{\GroupMemberThree}\par
            }
            \vspace{20pt}
        }
        \begin{abstract}
        % 6. Fill in your abstract    
        	The Oregon State University (OSU) Veterinary Hospital is need of a new software communication system to help manage their daily operations and improve the workflow of the business. An updated communication system will provide an environment for employees to manage and complete their tasks with better precision and speed. It will eliminate a waste of both paper and time as well as allowing the hospital to track goals on response times. Our group working with the OSU Veterinary Hospital will create a new communication system that will help to improve the hospital’s performance and ability to communicate information with owners.
        \end{abstract}     
    \end{singlespace}
\end{titlepage}
\newpage
\pagenumbering{arabic}
\tableofcontents
% 7. uncomment this (if applicable). Consider adding a page break.
%\listoffigures
%\listoftables
\clearpage

% 8. now you write!
\section{Problem Defninition}
Having an efficient and easy-to-use software system for managing operations is integral for any company, and the Oregon State University (OSU) Veterinary Hospital, a primary veterinary referral hospital in Oregon, is no exception. The current software used by the hospital to manage their daily operations is decades old and doesn’t efficiently work to manage appointments, track patients, or communicate within the hospital and with outside Veterinarians. This software causes many unnecessary errors and patient and owner suffering that could be avoided with a better system. The hospital needs a new software package that will not only provide a system that doesn’t hinder workflow, but will improve it and allow for increased efficiency. There are generic products that are currently available on the market, but none of them are flexible enough to meet the needs of the hospital, so a customized system is needed to solve their problems.

One flaw of the current system from a workflow perspective is that it is owner focused and not patient focused. Hospital employees would like to be able to look at records or history for a certain patient, but instead can only see it by an owner, who might have multiple pets. The current process is very slow and manual, with lots of movement around a large hospital and unnecessary printing of papers. For instance, if an owner calls the hospital with a medical question, the receptionist would log the message into the computer, print it out, and physically walk the paper across the hospital to put in a spinning queue. When a doctor or other staff member looks at it, they would write an answer or necessary action on the paper, and physically carry the paper to the next location. Eventually, after it reaches all the necessary staff, it would be returned to the receptionist, who would then call the owner back to provide them with the information they requested, whether that is an answer to a question or to schedule an appointment.  This process is not only inefficient but leaves room for error with losing pieces of paper or misreading handwritten notes. Another issue lies with the appointment scheduling system, as it can’t be seen in the current system whether a patient has an appointment already scheduled or not.


\section{Proposed Solution}
To help the hospital meet its goals and improve the way staff communicates in the hospital, a new communication system is essential. Our team will create, conceptualize, and develop a more current hospital communication system to be integrated with the current system that will streamline scheduling appointments and internal hospital communications. By mirroring the ideal workflow and the way messages are moved throughout the hospital, we will create a communication system where staff will be able to choose where a message is routed and assign a category, such as billing or medical, and it will be put in a queue for the necessary staff to look at when they have a chance. Then, notes can be added, and it can be rerouted to a new location. When the message gets back to reception or its final location, the whole chain of where the message has been will be visible. This will help eliminate some of the paper waste as well as reduce the amount of errors in the system and time spent running the papers around the hospital.

In addition to the improved communication, we will provide reports for the hospital to analyze different aspects of the communication process, such as response time for phone calls, so they can track how well they are meeting their goal of 24-hour response time to calls.  The new system will also provide the hospital with the ability to easily track and find information about patients and take a patient centric approach as opposed to the current owner centered one. We will  implement this in the form of a patient dashboard that will have easy access to important information about a pet, such as which vaccines an animal has previously been given, the dates of previous appointments, and dates of appointments scheduled for the future.

A final feature of the system would be to allow for employees to have profiles where they could be assigned to departments that would determine what information was available to them. Then if they needed access to different information for a day or because they changed jobs, their profile could be changed and the information they needed would be available to them.


\section{Performance Metrics}
There are several metrics that will help us determine if the project is a success based upon the desired features of the product. The most important metric is that messages must be able to be tracked and there must be a way to view the entire chain of where the message has been routed to along with all the notes added to the message. Next, it is important that the system will be able to assign a category to a message taken down by reception. The system must also allow for searching or filtering by category to only show messages that are a part of a given category. Another metric is that the system will allow for searching for information by a given pet. This will bring up a history of all messages created for that pet to be viewed or edited. The system must be able to allow the addition of future appointments so that it can easily be seen when an appointment has already been scheduled. Last, there needs to be profiles that can be assigned to employees and edited to allow them to see the information that is necessary to complete their jobs.

The team will be developing while in constant communication with the client and contacts at the hospital. Should the wishes of the clients change, or development is restricted based on technologies or other factors, these metrics may change. In addition, should the hospital request features or additions that cannot be completed during this year for whatever reason, the development team will document and prepare the project for any future development work by another team.


\end{document}