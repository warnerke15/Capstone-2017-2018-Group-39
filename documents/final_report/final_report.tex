\documentclass[onecolumn, draftclsnofoot,10pt, compsoc]{IEEEtran}
\usepackage{graphicx}
\usepackage{url}
\usepackage{setspace}

\usepackage{listings}
\usepackage{xcolor}
\usepackage{float}

\usepackage{geometry}
\geometry{textheight=9.5in, textwidth=7in}

\usepackage{hyperref}
\hypersetup{
    colorlinks=true,
    linkcolor=blue,
    filecolor=magenta,      
    urlcolor=cyan,
}

\makeatletter
\renewcommand\paragraph{\@startsection{paragraph}{4}{\z@}%
            {-2.5ex\@plus -1ex \@minus -.25ex}%
            {1.25ex \@plus .25ex}%
            {\normalfont\normalsize\bfseries}}
\makeatother
\setcounter{secnumdepth}{4} % how many sectioning levels to assign numbers to
\setcounter{tocdepth}{4}    % how many sectioning levels to show in ToC


\definecolor{lightgray}{rgb}{.9,.9,.9}
\definecolor{darkgray}{rgb}{.4,.4,.4}
\definecolor{purple}{rgb}{0.65, 0.12, 0.82}

\lstdefinelanguage{JavaScript}{
  keywords={typeof, new, true, false, catch, function, return, null, catch, switch, var, if, in, while, do, else, case, break},
  keywordstyle=\color{blue}\bfseries,
  ndkeywords={class, export, boolean, throw, implements, import, this},
  ndkeywordstyle=\color{darkgray}\bfseries,
  identifierstyle=\color{black},
  sensitive=false,
  comment=[l]{//},
  morecomment=[s]{/*}{*/},
  commentstyle=\color{purple}\ttfamily,
  stringstyle=\color{red}\ttfamily,
  morestring=[b]',
  morestring=[b]"
}
 
\urlstyle{same}

% 1. Fill in these details
\def \CapstoneTeamName{		ConnectBasket Development Team}
\def \CapstoneTeamNumber{		39}
\def \GroupMemberOne{			Henry Fowler}
\def \GroupMemberTwo{			Kailyn Hellwege}
\def \GroupMemberThree{			Taylor Kirkpatrick}
\def \CapstoneProjectName{		ConnectBasket}
\def \CapstoneSponsorCompany{	OSU MIME}
\def \CapstoneSponsorPerson{		Dr. Chinweike Eseonu}

% 2. Uncomment the appropriate line below so that the document type works
\def \DocType{		%Problem Statement
				%Requirements Document
				%Technology Review
				%Design Document
				Final Report
				}
			
\newcommand{\NameSigPair}[1]{\par
\makebox[2.75in][r]{#1} \hfil 	\makebox[3.25in]{\makebox[2.25in]{\hrulefill} \hfill		\makebox[.75in]{\hrulefill}}
\par\vspace{-12pt} \textit{\tiny\noindent
\makebox[2.75in]{} \hfil		\makebox[3.25in]{\makebox[2.25in][r]{Signature} \hfill	\makebox[.75in][r]{Date}}}}
% 3. If the document is not to be signed, uncomment the RENEWcommand below
\renewcommand{\NameSigPair}[1]{#1}

%%%%%%%%%%%%%%%%%%%%%%%%%%%%%%%%%%%%%%%
\begin{document}
\begin{titlepage}
    \pagenumbering{gobble}
    \begin{singlespace}
    	\includegraphics[height=4cm]{coe_v_spot1}
        \hfill 
        % 4. If you have a logo, use this includegraphics command to put it on the coversheet.
        %\includegraphics[height=4cm]{CompanyLogo}   
        \par\vspace{.2in}
        \centering
        \scshape{
            \huge CS Capstone \DocType \par
            {\large\today}\par
            \vspace{.5in}
            \textbf{\Huge\CapstoneProjectName}\par
            \vfill
            {\large Prepared for}\par
            \Huge \CapstoneSponsorCompany\par
            \vspace{5pt}
            {\Large\NameSigPair{\CapstoneSponsorPerson}\par}
            {\large Prepared by }\par
            Group\CapstoneTeamNumber\par
            % 5. comment out the line below this one if you do not wish to name your team
            \CapstoneTeamName\par 
            \vspace{5pt}
            {\Large
                \NameSigPair{\GroupMemberOne}\par
                \NameSigPair{\GroupMemberTwo}\par
                \NameSigPair{\GroupMemberThree}\par
            }
            \vspace{20pt}
        }
        \begin{abstract}
        % 6. Fill in your abstract    
        

        \end{abstract}     
    \end{singlespace}
\end{titlepage}
\newpage
\pagenumbering{arabic}
\tableofcontents
% 7. uncomment this (if applicable). Consider adding a page break.
%\listoffigures
%\listoftables
\clearpage

% 8. now you write!

\section{Introduction to Project}

\section{Requirements Document}

\section{Design Document}

\section{Tech Review}

\section{Weekly Blog Posts}

\subsection{Henry}

\subsection{Kailyn}

\subsection{Taylor}

\section{Final Poster}


\section{Project Documentation}


\section{Recommended Technical Resources for Learning More}
\subsection {Helpful Websites}
\begin{itemize}
\item https://docs.angularjs.org/guide/introduction
\item https://www.w3schools.com/
\item http://php.net/
\item https://dev.mysql.com/doc/
\end{itemize}


\section{Conclusions and Reflections}

\subsection{Henry}

\subsubsection{What technical information did you learn?}

\subsubsection{What non-technical information did you learn?}

\subsubsection{What have you learned about project work?}

\subsubsection{What have you learned about project management?}

\subsubsection{What have you learned about working in teams?}

\subsubsection{If you could do it all over, what would you do differently?}

\subsection{Kailyn}

\subsubsection{What technical information did you learn?}

\subsubsection{What non-technical information did you learn?}

\subsubsection{What have you learned about project work?}

\subsubsection{What have you learned about project management?}

\subsubsection{What have you learned about working in teams?}

\subsubsection{If you could do it all over, what would you do differently?}

\subsection{Taylor}

\subsubsection{What technical information did you learn?}

\subsubsection{What non-technical information did you learn?}

\subsubsection{What have you learned about project work?}

\subsubsection{What have you learned about project management?}

\subsubsection{What have you learned about working in teams?}

\subsubsection{If you could do it all over, what would you do differently?}



\section{Appendix 1: Essential Code Listings}

\section{Appendix 2: Photos of Project}


\end{document}
